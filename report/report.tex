% IEEE standard conference template; to be used with:
%   spconf.sty  - LaTeX style file, and
%   IEEEbib.bst - IEEE bibliography style file.
% --------------------------------------------------------------------------

\documentclass[letterpaper]{article}
\usepackage{spconf,amsmath,amssymb,graphicx}

% Example definitions.
% --------------------
% nice symbols for real and complex numbers
\newcommand{\R}[0]{\mathbb{R}}
\newcommand{\C}[0]{\mathbb{C}}

% bold paragraph titles
\newcommand{\mypar}[1]{{\bf #1.}}

% Title.
% ------
\title{Implementation of an optimized authenticated encryption scheme}
%
% Single address.
% ---------------
\name{Alessandro Giaconia, Lorenzo Paleari, Diana Khimey, Filippo Visconti}
\address{Department of Computer Science\\ ETH Z\"urich\\Z\"urich, Switzerland}

% For example:
% ------------
%\address{School\\
%		 Department\\
%		 Address}
%
% Two addresses (uncomment and modify for two-address case).
% ----------------------------------------------------------
%\twoauthors
%  {A. Author-one, B. Author-two\sthanks{Thanks to XYZ agency for funding.}}
%		 {School A-B\\
%		 Department A-B\\
%		 Address A-B}
%  {C. Author-three, D. Author-four\sthanks{The fourth author performed the work
%		 while at ...}}
%		 {School C-D\\
%		 Department C-D\\
%		 Address C-D}
%

\begin{document}
%\ninept
%
\maketitle
%

% The hard page limit is 6 pages in this style. Do not reduce font size
% or use other tricks to squeeze. This pdf is formatted in the American letter format, so the spacing may look a bit strange when printed out.

\begin{abstract}
% Describe in concise words what you do, why you do it (not necessarily
% in this order), and the main result.  The abstract has to be
% self-contained and readable for a person in the general area.
You should write the abstract last.
\end{abstract}

\section{Introduction}\label{sec:intro}

% Do not start the introduction with the abstract or a slightly modified
% version. It follows a possible structure of the introduction.
% Note that the structure can be modified, but the
% content should be the same. Introduction and abstract should fill at most the first page, better less.

\mypar{Motivation}
% The first task is to motivate what you do.  You can
% start general and zoom in one the specific problem you consider.  In
% the process you should have explained to the reader: what you are doing,
% why you are doing, why it is important (order is usually reversed).
%
% For example, if my result is the fastest sorting implementation ever, one
% could roughly go as follows. First explain why sorting is important
% (used everywhere with a few examples) and why performance matters (large datasets,
% realtime). Then explain that fast implementations are very hard and
% expensive to get (memory hierarchy, vector, parallel).
%
% Now you state what you do in this paper. In our example:
% presenting a sorting implementation that is
% faster for some sizes as all the other ones.
In an era dominated by the exponential growth of data and the increasing prevalence of digital communication,
the imperative to secure sensitive information has never been more crucial.
As technology advances, so do the challenges in maintaining the confidentiality and integrity of data exchanged over networks.
This project embarks on a journey through the implementation of a cutting-edge,
optimized, high-performance authenticated encryption scheme.
The motivation behind this endeavor stems from the need for robust cryptographic solutions, which still achieve good runtimes.
Authenticated encryption schemes are a fundamental building block of modern cryptography, which
are used to provide confidentiality, integrity and authenticity of data.
Their use is widespread in many applications, such as TLS, SSH, IPsec, and many others.
Leveraging the combination of ChaCha20 and Blake3, our project aims to prioritize efficiency and speed.

The key difference between authenticated encryption schemes and encryption schemes
is that the former also provide integrity and authenticity of data, while the latter only provides confidentiality.



\mypar{Related work} Next, you have to give a brief overview of
related work. For a report like this, anywhere between 2 and 8
references. Briefly explain what they do. In the end contrast to what
you do to make now precisely clear what your contribution is.

\section{Background: Whatever the Background is}\label{sec:background}
In this section, we give a brief overview of the background necessary to understand the rest of the report.
In particular, we introduce the concepts of encryption and authentication, and we explain the algorithms we use.
% Give a short, self-contained summary of necessary
% background information. For example, assume you present an
% implementation of sorting algorithms. You could organize into sorting
% definition, algorithms considered, and asymptotic runtime statements. The goal of the
% background section is to make the paper self-contained for an audience
% as large as possible. As in every section
% you start with a very brief overview of the section. Here it could be as follows: In this section
% we formally define the sorting problem we consider and introduce the algorithms we use
% including a cost analysis.

\mypar{Encryption}
% Precisely define sorting problem you consider.

\mypar{Authentication}
Authentication is the process through which the integrity and origin of data are verified, to ensure that
is has not been tampered with, and to confirm that it was sent by the expected sender.
This process is critical in various fields, such as secure communication, electronic commerce, and many others.

In order to achieve authentication, a Message Authentication Code is usually used. A MAC is a short piece of information
that is used to authenticate a message, in combination with a secret key. The receiver of the message can verify the authenticity
of the message by recomputing the MAC and comparing it with the original MAC.

% As an aside, don't talk about "the complexity of the algorithm.'' It's incorrect,
% problems have a complexity, not algorithms.
\mypar{Blake3}
We chose Blake3 for the authentication part of the scheme, due to its high performance and security guarantees.
By design, Blake3 is also highly parallelizable, which makes it a good candidate for our implementation.
For the implementation, we used the C language and two different approaches:
one implementation is using a stack, which is more versatile,
and the other one uses a more efficient divide-and-conquer approach, which is a bit less flexible, despite performing much better.
Blake3 is a cryptographic hash function which also provides a MAC function, the one we're interested in.
At a high level, Blake3 works by splitting the input into chunks of size up to 1024 bytes, and then compressing them using a compression function, which
performs a series of operations on the input, and returns a 64B output, named the chaining value.
The chaining value is then used as input for the next chunk, and the process is repeated until all the chunks have been processed.
These chunks form the leaves of a binary tree, which is traversed from the bottom up, until the root is reached, by combining the chaining values of the children nodes, two at a time.
After a number of iterations, we're left with a single node, the root node, whose chaining value is the output of the algorithm.
By repeating the last call to the compression function with an increased counter, we can obtain an output of arbitrary length.

\section{Your Proposed Method}\label{sec:yourmethod}
In this section, we dive into the details of our implementation. We describe how we implemented the algorithms,
and we explain the optimizations we used to achieve high performance.
% Now comes the ``beef'' of the report, where you explain what you
% did. Again, organize it in paragraphs with titles. As in every section
% you start with a very brief overview of the section.
%
% In this section, structure is very important so one can follow the technical content.
%
\mypar{Blake3}
In this subsection, we describe our implementations of the Blake3 algorithm, which are based on and closely follow the original Blake3 paper that describes the algorithm in detail.
Given their different strenghts and characteristics, we also included a dispatcher that chooses the best implementation based on the input size, to always guarantee a correct output.
\mypar{Stack version}
TODO

\mypar{Divide-and-conquer version}
The divide-and-conquer version of the algorithm is based on the assumption that all of our input is available at the beginning of the computation.
This allows us to evenly distribute the work among the threads, which have no dependencies on each other, since each thread works on a different portion of the input.
The threads work in parallel until they reach the base case, which is when the input size is small enough to be processed sequentially.
At this point, the threads return the result to the parent thread, which combines them and returns the final result.
\begin{figure}[h]
  \begin{center}
    \includegraphics[width=0.45\textwidth]{figures/div_conq.png}
  \end{center}
  \caption{High-level visualization of the divide and conquer approach}\label{fig:divide_and_conquer}
\end{figure}

To achieve this, we use the OpenMP library, which allows us to easily parallelize the computation once we have created the parallel region.
In addition to that, we vectorized the compression function, which is the most computationally intensive part of the algorithm, using the AVX instruction set and the code available on GitHub from the reference implementation.

The main advantage of this approach is that it's very scalable, since the work is evenly distributed among the threads, and the threads have no dependencies on each other. This allows us to achieve very good speedups, as we will see in the results section.
The main disadvantages are that it's not very flexible, since it requires all of the input to be available at the beginning of the computation (although this is not a limitation for us), and that is less flexible, as, to achieve its maximum potential,
it requires the input to be full trees.
TODO Mention and cite any external resources that you used including libraries or other code.

\section{Experimental Results}\label{sec:exp}

Here you evaluate your work using experiments. You start again with a
very short summary of the section. The typical structure follows.

\mypar{Experimental setup} Specify the platform (processor, frequency, maybe OS, maybe cache sizes)
as well as the compiler, version, and flags used. If your work is about performance,
I strongly recommend that you play with optimization flags and consider also icc for additional potential speedup.

Then explain what kind of benchmarks you ran. The idea is to give enough information so the experiments are reproducible by somebody else on his or her code.
For sorting you would talk about the input sizes. For a tool that performs NUMA optimization, you would specify the programs you ran.

\mypar{Results}
Next divide the experiments into classes, one paragraph for each. In each class of experiments you typically pursue one questions that then is answered by a suitable plot or plots. For example, first you may want to investigate the performance behavior with changing input size, then how your code compares to external benchmarks.

For some tips on benchmarking including how to create a decent viewgraph see pages 22--27 in \cite{Pueschel:10}.

{\bf Comments:}
\begin{itemize}
\item Create very readable, attractive plots (do 1 column, not 2 column plots
for this report) with readable font size. However, the font size should also not be too large; typically it is smaller than the text font size.
An example is in Fig.~\ref{fftperf} (of course you can have a different style).
\item Every plot answers a question. You state this question and extract the
answer from the plot in its discussion.
\item Every plot should be referenced and discussed.
\end{itemize}

\begin{figure}\centering
  \includegraphics[scale=0.33]{dft-performance.eps}
  \caption{Performance of four single precision implementations of the
  discrete Fourier transform. The operations count is roughly the
  same. The labels in this plot are maybe a little bit too small.\label{fftperf}}
\end{figure}

\section{Conclusions}

Here you need to summarize what you did and why this is
important. {\em Do not take the abstract} and put it in the past
tense. Remember, now the reader has (hopefully) read the report, so it
is a very different situation from the abstract. Try to highlight
important results and say the things you really want to get across
such as high-level statements (e.g., we believe that .... is the right
approach to .... Even though we only considered x, the
.... technique should be applicable ....) You can also formulate next
steps if you want. Be brief. After the conclusions there are only the references.

\section{Further comments}

Here we provide some further tips.

\mypar{Further general guidelines}

\begin{itemize}
\item For short papers, to save space, I use paragraph titles instead of
subsections, as shown in the introduction.

\item It is generally a good idea to break sections into such smaller
units for readability and since it helps you to (visually) structure the story.

\item The above section titles should be adapted to more precisely
reflect what you do.

\item Each section should be started with a very
short summary of what the reader can expect in this section. Nothing
more awkward as when the story starts and one does not know what the
direction is or the goal.

\item Make sure you define every acronym you use, no matter how
convinced you are the reader knows it.

\item Always spell-check before you submit (to us in this case).

\item Be picky. When writing a paper you should always strive for very
high quality. Many people may read it and the quality makes a big difference.
In this class, the quality is part of the grade.

\item Books helping you to write better: \cite{Higham:98} and \cite{Strunk:00}.

\item Conversion to pdf (latex users only):

dvips -o conference.ps -t letter -Ppdf -G0 conference.dvi

and then

ps2pdf conference.ps
\end{itemize}

\mypar{Graphics} For plots that are not images {\em never} generate the bitmap formats
jpeg, gif, bmp, tif. Use eps, which means encapsulate postscript. It is
scalable since it is a vector graphic description of your graph. E.g.,
from Matlab, you can export to eps.

The format pdf is also fine for plots (you need pdflatex then), but only if the plot was never before in the format
jpeg, gif, bmp, tif.


% References should be produced using the bibtex program from suitable
% BiBTeX files (here: bibl_conf). The IEEEbib.bst bibliography
% style file from IEEE produces unsorted bibliography list.
% -------------------------------------------------------------------------
\bibliographystyle{IEEEbib}
\bibliography{bibl_conf}

\end{document}
